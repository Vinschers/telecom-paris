\documentclass{article}

\usepackage{geometry}
\geometry{margin=1in}

\usepackage{amsmath}

\newcounter{problem}
\newcounter{solution}

\newcommand\Problem{%
  \stepcounter{problem}%
  \textbf{\theproblem.}~%
  \setcounter{solution}{0}%
}

\newcommand\Solution{%
  \textbf{Solution:}\\%
}

\title{Image: TP 3}
\author{Felipe Scherer Vicentin}

\begin{document}
\maketitle

\section{Détection de contours}
\subsection{Filtre de gradient local par masque}

\Problem{Rappelez l'intérêt du filtre de Sobel, par rapport au filtre différence, qui calcule une dérivée par la simple différence entre deux pixels voisins.}

\Solution{The Sobel filter automatically applies a low-pass filter before calculating the gradient. This makes the contours more stable when the image has noise or blur.}

\Problem{Est-il nécessaire de faire un filtre passe-bas de l’image avant d’utiliser le filtre de Sobel?}

\Solution{No, since its kernel is built to already compute the low-pass filter before the gradient calculation.}

\Problem{Le seuillage de la norme du gradient permet d'obtenir des contours.
Commentez la qualité des contours obtenus (robustesse au bruit, continuité, épaisseur, position, \ldots) quand l'on fait varier ce seuil.}

\Solution{The threshold applied to the gradient's norm can change how much information is retained in the contours.
If the threshold is too low, then noise and regions that are not edges may be counted in the conteurs.
If it is too big, then some edges are lost.
In general, the smaller the threshold, the thicker the conteurs.
The position of the contours, does not change though.}

\subsection{Maxima du module du gradient dans la direction du gradient}

\Problem{Rappeler la formule qui donne les points sélectionnés par cette fonction (cours)?}

\Solution{First, we select the points whose gradient's norm is at least $\sigma_1$:
$E_1 = \{\mathbf{x} \colon \vert \nabla I(\mathbf{x}) \vert \geq \sigma_1 \}$.
Then, we filter $E_1$ by keeps the points that are connected to points whose gradient's norm is at least $\sigma_2$:
$E_2 = \{\mathbf{x} \colon \vert \nabla I(\mathbf{x}) \vert \geq \sigma_2 \}$.}

\Problem{Il est possible d'éliminer les contours dont la norme du gradient est inférieure à un seuil donné.
Commentez les résultats obtenus en terme de position et de continuité des contours, et de robustesse au bruit en faisant varier ce seuil.}

\Solution{The standard threshold used is $0.0001$. With such a threshold, most of the resulting image is comprised by noise and outliers that are not edges.
By varing this parameter, we can filter out the noise and highlight the actual contours.}

\Problem{Cherchez à fixer le seuil sur la norme de façon à obtenir un compromis entre robustesse au bruit et continuité des contours.}

\Solution{It seems that a threshold of $0.2$ yields a good result, highlighting the edges and filtering out the noise of the image.}

\subsection{Passage par zéro du laplacien}

\Problem{Quel est l'effet du paramètre $\alpha$ sur les résultats?}

\Solution{The bigger the parameter $\alpha$, the greater the strength of the low-pass filter.}

\Problem{Sur l'image \texttt{cell.tif}, quelles sont les principales différences par rapport aux résultats fournis par les opérateurs vus précédemment?}

\Solution{This segmentatios appears to leave the result with more noise. It also seems to highlight the interior of cells, instead of just their edges.}

\Problem{Sur l'image \texttt{pyramide.tif}, comment est-il possible de supprimer les faux contours créés par cette approche?}

\Solution{A simple solution is to increaste the threshold of the gradient ``lpima''.
We can also increase the value of $\alpha$ since it reduces the smoothness of the image and, thus, makes the contours sharper.}

\subsection{Adaptation en fonction de l'image}

\Problem{Quel opérateur choisiriez-vous pour segmenter l'image \texttt{pyra-gauss.tif}?}

\Solution{I would choose the sobel operator.}

\Problem{Quels seraient les pré-traitements et les post-traitements à effectuer?}

\Solution{For the pre-processing, using a low-pass filter to further decrease the presence of high-frequency noise makes the result better.
For the post-processing, applying a threshold to the gradient is enough to correctly highlight the edges of the image.}

\section{Segmentation par classification: K-moyennes}

\subsection{Image à niveaux de gris}

\Problem{Testez l'algorithme des k-moyennes sur l'image \texttt{cell.tif} pour une classification en 2 classes.
Cette classification segmente-t-elle correctement les différents types de cellules? Si non, que proposez-vous?}

\Solution{The segmentation is not completely correct.
Many cells whose centre is lighter end up in a different class.
To solve this, we could apply an enclosing morphological operator in the image before running the K-means algorithm.}

\Problem{Testez les différentes possibilités pour initialiser les classes.
Décrivez si possible ces différentes méthodes.}

\Solution{If the random state is removed from the KMeans initialization, the initial positions of the clusters will change in each run of the algorithm.
Indeed, depending on how the initial centroids are placed, the K-means algorithm may fail to identify the cells and the background.
For instance, if the two initial centroids end up in extremely different colors, the cells are not detected at all.}

\Problem{La classification obtenue est-elle stable (même position finale des centres des classes) avec une initialisation aléatoire? Testez sur différentes images à niveaux de gris et différents nombres de classes.}

\Solution{The classification is not stable.
Depending on the initial centroids and the number of classes, the result may change substantially.}

\Problem{Quelles sont les difficultés rencontrées pour la segmentation des différentes fibres musculaires dans l'image \texttt{muscle.tif}?}

\Solution{The simple classification based only on colors cannot correctly identify the fibers, since there is a lot of white parts in the cells.}

\Problem{Expliquez pourquoi le filtrage de l'image originale (filtre de la moyenne ou filtre median) permet d'améliorer la classification.}

\Solution{teste}

\subsection{Image en couleur}

\Problem{Testez l'algorithme sur l'image \texttt{fleur.tif} pour une classification en 10 classes, les centres des classes initiaux étant tirés aléatoirement.}

\Solution{teste}

\Problem{Commentez la dégradation de l'image quantifiée par rapport à l'image initiale.}

\Solution{teste}

\Problem{Quel est le nombre minimum de classes qui donne un rendu visuel similaire à celui de l'image codée sur 3 octets?}

\Solution{teste}

\section{Seuillage automatique: Otsu}

\Problem{Dans le script \texttt{otsu.py} quel critère cherche-t-on à optimiser?}

\Solution{teste}

\Problem{Testez la méthode de Otsu sur différentes images à niveaux de gris, et commentez les résultats.}

\Solution{teste}

\Problem{Cette méthode permet-elle de seuiller correctement une image de norme du gradient?}

\Solution{teste}

\Problem{Modifiez ls script \texttt{otsu.py} pour traiter le problème à trois classes, i.e., la recherche de deux seuils.}

\Solution{teste}

\end{document}
